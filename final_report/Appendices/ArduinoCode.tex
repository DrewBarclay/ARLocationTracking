% ArduinoCode

\chapter{Source Code} % Main appendix title
\label{SourceCode} % For referencing this appendix elsewhere, use \ref{AppendixA}

This appendix lists all the code used in the project. The project names are named according to their function.

\section{AR Code}
All of the Android application code is packaged in \code{ARLocationTracking}. It also contains code to handle the rendering
of pointers and texts on the screen, as well as code allowing the phone to communicate
with the Arduino chips via FTDI and USB. A modified version of the Texample23D open source project, used for rendering text in 3D, is included in the project. The modifications allow the 3D text to be rotated with an arbitrary matrix. This is used to billboard the rendered text. Finally, the EAGLE schematics for the tags' PCB is contained within this project. The project can be found at \url{https://github.com/DrewBarclay/ARLocationTracking}.

\section{Arduino Code}
In \code{dw1000arduinonetwork} is all of the code that runs on the Arduinos. It contains all of the code necessary
to interface with the DWM1000 and form a network. This package does not contain any position calculation code; that can be found in \code{TagParser.java} in the AR Code. The project includes a modified version of the \code{arduino-dw1000} project by Thomas Trojer. The modifications allow the easy setting of antenna delay from the main Arduino code. The code can be opened and compiled in Arduino IDE. Arduino IDE must be configured to compile C++ code. The project can be found at \url{https://github.com/DrewBarclay/dw1000arduinonetwork}.

