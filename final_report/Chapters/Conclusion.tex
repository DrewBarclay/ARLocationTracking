% Conclusion

\chapter{Conclusion} % Main chapter title

\label{Conclusion} % For referencing the chapter elsewhere, use \ref{Chapter1} 

%----------------------------------------------------------------------------------------

Lastly, the our project shows that with much more refinement, a practical application of augmented reality is closer to reality than previously envisioned. This is because it is operating on a device that can be easily obtained by the average consumer. While devices such as the HTC Vive and Occulus Rift demonstrate the power of virtual reality, and the potential to be augmented in real life, it is currently out of the practical price range of many consumers. Most of the time, a user will also require a powerful personal computer to run the programs, which can range at least one thousand dollars. Google Cardboard shows that with an Android device, that this is subjectively cheaper, where the tags and chips must be obtained on top of the Android device. There are three main topics that were covered in the report: augmented reality rendering, position calculation and range finding.

Most of the data is processed on our Android devices, read from tags attached to an Arduino chip. Using 3D space equations to translate the data to a 2D plane, allows data processed from the chips. The phones then render the graphical overlay in real time with a live video feed to show the direction of other devices or tags in the network. (to edit later if we put in images with the complete HUD: The report shows an demonstration of what the screen can be.) There will be a live demonstration in the final presentation.

Locating devices and tags are done with the tags to send directly to the Android device. It is done UWB to communicate between each part of the network. A minimum of four tags are necessary to determine a user’s distance and direction to his destination in a 3D plane. The more tags available, the more accurate the reading will be. This is essentially the bread and butter of the project; the device on which this can be processed and rendered on does not matter. Without the tags, there would be no data to be drawn about a user’s location and the destination.

The realization of a HUD becoming a real and practical application to everyday lives can be a fascinating endeavor. It cannot be limited to science fantasy and video games no longer. The applications of this project are astounding and very broad. It can range from surgeons knowing the status of their patient, location tracking, entertainment. The amount is limitless!

Pokemon GO! was the start of many augmented reality applications available on Google Play. The hope is that other developers realize the potential of Augmented Reality and not only make it accessible to normal, everyday citizens but to also improve it to such a point where it will be applied as part of a more accessible everyday accessory such as watches and glasses.

To have all the information ready at your fingertips, or shall we say, in front of your face is a blessing to behold. It allows for the quicker passing of information quickly and received in such a fast and direct manner. If one obtains and processes of the situation at hand, he shall receive the edge. To him, he shall be the victor.
