% Conclusion

\chapter{Conclusion} % Main chapter title

\label{Conclusion} % For referencing the chapter elsewhere, use \ref{Chapter1} 

%----------------------------------------------------------------------------------------

The project shows that with much more refinement, a practical application of augmented reality is closer to reality than previously envisioned. The project is operating on a device that can be easily obtained by the average consumer. While devices such as the HTC Vive and Occulus Rift demonstrate the power of virtual reality and the potential to be augmented in real life, it is currently out of the practical price range of many consumers. Most of the time, a user will also require a powerful personal computer to run the programs, which can range at least one thousand dollars. Google Cardboard shows that with an Android device virtual reality can be experienced more cheaply, except for the tags which must be obtained on top of the Android device. 

There are three main topics that were covered in this report: augmented reality rendering, position calculation and range finding. Distances between devices are determined via ultra-wideband transceivers using time-of-flight calculations. These distances are then used to calculate positions of the objects in 3D space. Finally, these positions are rendered on the screen with the augmented reality subsystem.

The system is able to accurately display the locations of tracked objects to under 50 cm at distances over 10 meters, and provide updates to their positions at over 4Hz.

A HUD has been realized outside of videogamesl. The applications of the techniques in this report are numerous. They can range from surgeons knowing the status of their patient, location tracking, and more immersive entertainment. 

\section{Future Work}
A number of aspects of work on the project were left incomplete due to time constraints. They offer paths for future work to focus on:

\begin{enumerate}
	\item An integrated GPS location could be included that would display the locations of GPS coordinates on the display relative to your current GPS coordinates.
	\item The Arduino Pro Mini could be upgraded with a faster microcontroller which could improve the system operating frequency by an order of magnitude.
	\item The anchors could have PCBs designed for them.
	\item An automatic calibration system could be designed that such that the user does not have to manually calibrate the system on startup. The system would use dead reckoning and the velocity of the user as measured by the position calculation system to determine a rotation matrix to calibrate the system.
	\item A more powerful mathematical technique for calculating positions in 3D space could be used which takes in the extra range data that the tags have to each other. This could be used to minimize error and filter out noise.
\end{enumerate}